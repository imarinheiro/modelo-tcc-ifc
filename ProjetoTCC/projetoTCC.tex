\documentclass[12pt,article,a4paper,brazil,oldfontcommands,oneside]{abntex2}
\usepackage{lmodern}			% Usa a fonte Latin Modern
\usepackage[T1]{fontenc}		% Selecao de codigos de fonte.
\usepackage[utf8]{inputenc}		% Codificacao do documento (conversão automática dos acentos)
\usepackage{indentfirst}		% Indenta o primeiro parágrafo de cada seção.
\usepackage{nomencl} 			% Lista de simbolos
\usepackage{color}				% Controle das cores
\usepackage{graphicx}			% Inclusão de gráficos
\usepackage{microtype} 			% para melhorias de justificação
\usepackage{rotating}
\usepackage[alf]{abntex2cite}	% Citações padrão ABNT

% Configuração de Margens e espaços
\setlrmarginsandblock{3cm}{3cm}{*}
\setulmarginsandblock{3cm}{3cm}{*}
\setlength{\ABNTEXcitacaorecuo}{1.8cm}
\checkandfixthelayout
\setlength{\parindent}{1.3cm}			% O tamanho do parágrafo é dado por:
\setlength{\parskip}{0.2cm}  			% Controle do espaçamento entre um parágrafo e outro:
%\SingleSpacing

% Informações de dados para CAPA e FOLHA DE ROSTO
\titulo{Modelo de Projeto de TCC do \\ IFC - Araquari usando \abnTeX}
\autor{Nome do Aluno \and Orientador: Nome do Orientador}
\local{Araquari -- SC -- Brasil}
\data{Setembro de 2015}

% Configurações de aparência do PDF final
\makeatletter
\hypersetup{
     	%pagebackref=true,
		pdftitle={\@title}, 
		pdfauthor={\@author},
    	pdfsubject={Modelo de projeto e TCC},
	    pdfcreator={LaTeX with abnTeX2},
		pdfkeywords={abnt}{latex}{abntex2}{modelo}{TCC}, 
		colorlinks=true,       		% false: boxed links; true: colored links
    	linkcolor=blue,          	% color of internal links
    	citecolor=blue,        		% color of links to bibliography
    	filecolor=magenta,      	% color of file links
		urlcolor=blue,
		bookmarksdepth=4
}
\makeatother

% compila o indice
\makeindex


% Início do documento
\begin{document}

\frenchspacing 		% Retira espaço extra obsoleto entre as frases.
\maketitle			% página de titulo


\section{Tema}

O tema é a parte geral da pesquisa; é amplo, porém claro e objetivo, seguido da delimitação do tema, que deve ser ``funilado'', isto porque o pesquisador deve terminar seu trabalho em prazo determinado. Não delimitando o tema, o trabalho ficará muito amplo, o que dificultará seu término{marconi}.

\subsection{Delimitação do Tema}

O tema é a parte geral da pesquisa; é amplo, porém claro e objetivo, seguido da delimitação do tema, que deve ser ''funilado'', isto porque o pesquisador deve terminar seu trabalho em prazo determinado. Não delimitando o tema, o trabalho ficará muito amplo, o que dificultará seu término.

Alguns exemplos delimitados:

\begin{enumerate}
\item gravidez na adolescência: um estudo demonstrativo da gravidez entre as adolescentes de 14 a 18 anos na cidade de Araquari - Santa Catarina;
\item usuários de drogas: o que leva os adolescentes do sexo masculino entre 18 e 25 anos a fazerem uso de drogas na região Nordeste de Santa Catarina;
\end{enumerate}

\section{Problema}

O problema é a mola propulsora de todo o trabalho de pesquisa. Depois de definido o tema, levanta-se uma questão para ser respondida através de uma hipótese, que será confirmada ou negada através do trabalho de pesquisa. O Problema é criado pelo próprio autor e relacionado ao tema escolhido. O autor, no caso, criará um questionamento para definir a abrangência de sua pesquisa. Não há regras para se criar um Problema, mas alguns autores sugerem que ele seja expresso em forma de pergunta. Particularmente, prefiro que o Problema seja descrito como uma afirmação.

Segundo \citeonline[p. 220]{marconi} ``a formulação do problema prende-se ao tema proposto: ela esclarece a dificuldade específica com a qual se defronta e que se pretende resolver por intermédio da pesquisa''. 

\section{Justificativa}

É a parte em que o pesquisador expõe a relevância de seu trabalho quer na sociedade acadêmica, quer na sociedade científica ou na sociedade civil. Ele demonstrará a importância de seu trabalho e os benefícios que este trará à humanidade.

A Justificativa num projeto de pesquisa, como o próprio nome indica, é o convencimento de que o trabalho de pesquisa é fundamental de ser efetivado. O tema escolhido pelo pesquisador e a Hipótese levantada são de suma importância, para a sociedade ou para alguns indivíduos, de ser comprovada.

Deve-se tomar o cuidado, na elaboração da Justificativa, de não se tentar justificar a Hipótese levantada, ou seja, tentar responder ou concluir o que vai ser buscado no trabalho de pesquisa. A Justificativa exalta a importância do tema a ser estudado, ou justifica a necessidade imperiosa de se levar a efeito tal empreendimento. 

\section{Objetivos}
\subsection{Objetivo Geral do Trabalho}

A definição do Objetivo geral determina o que o pesquisador quer atingir com a realização do trabalho de pesquisa. Objetivo é sinônimo de meta, fim.

O objetivo geral pode ser um apanhado da problematização, até onde o pesquisador quer levar o seu estudo e a demonstração deste. Fazendo uso do mesmo tema, pode-se gerar o seguinte objetivo:

É importante lembrar que todos os objetivos, quer geral quer específicos, devem iniciar com verbo no infinitivo, assim: estudar, demonstrar, cultivar, proporcionar, especificar etc. Por exemplo: esclarecer tal coisa; definir tal assunto; procurar aquilo; permitir aquilo outro, demonstrar alguma coisa etc.

\subsection{Objetivos Específicos}

Os objetivos específicos podem ser entendidos como as tarefas que, se realizadas, levarão ao cumprimento do Objetivo Geral. São as etapas principais do Projeto.

Os objetivos específicos também serão baseados na problematização e têm a função de, juntos, levar à realização do objetivo geral, ou seja, o pesquisador, por meio dos objetivos específicos, chega ao objetivo geral. 

\section{Metodologia}

A Metodologia é a explicação minuciosa, detalhada, rigorosa e exata de toda ação desenvolvida no método (caminho) do trabalho de pesquisa. É a explicação do tipo de pesquisa, do instrumental utilizado (questionário, entrevista etc), do tempo previsto, da equipe de pesquisadores e da divisão do trabalho, das formas de tabulação e tratamento dos dados, enfim, de tudo aquilo que se utilizou no trabalho de pesquisa.

A metodologia é o conjunto de técnicas que o pesquisador utiliza para realizar seu trabalho. São aplicadas, dentre elas a observação direta intensiva, em que se inclui a própria observação e a entrevista. Há, também, a observação direta extensiva que, inclui o questionário, testes, análise de conteúdo, história de vida e pesquisa de mercado, dentre outras.

Exemplo:
\begin{enumerate}
\item Estudo de caso
\item Estudo comparativo
\item Pesquisa Teórica
\item Pesquisa Laboratorial
\item Aplicação
\item Avaliação
\item Avaliação de desempenho
\item Simulação
\item Questionários/Entrevistas 
\end{enumerate}


\section{Cronograma}

\begin{table}[h]
\centering
%\caption{Cronograma de atividades proposto}
\begin{tabular}{r||c|c|c|c|c|c} \hline
{\bf Atividades}        & {\bf Mês 1}	& {\bf  Mês 2} & {\bf  Mês 3} & {\bf  Mês 4} & {\bf  Mês 5}& {\bf  Mês 6} 
\\ \hline \hline
Levantamento do tema 	& \textbullet 	& 		& 		&	 	&		&	\\ \hline 
Pesquisa do tema	& \textbullet	& \textbullet	& \textbullet	&		&		&	\\ \hline
Atividade 1     	&	        & \textbullet	& \textbullet	& \textbullet	& \textbullet	& \textbullet \\ \hline
Atividade 2	        & 		& \textbullet	&		&		&	& \textbullet	\\ \hline
Conclusões		&		&		&		&		&	& \textbullet	\\ \hline
Apresentação		&		&		&		&		&	& \textbullet	\\ \hline
\end{tabular}
\label{tab:cronograma}
\end{table}


\bibliography{refs}
\end{document}
