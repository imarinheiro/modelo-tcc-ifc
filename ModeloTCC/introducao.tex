% ----------------------------------------------------------
% Introdução (exemplo de capítulo sem numeração, mas presente no Sumário)
% ----------------------------------------------------------
\chapter[Introdução]{Introdução}
%\addcontentsline{toc}{chapter}{Introdução}
% ----------------------------------------------------------

Este documento e seu código-fonte são exemplos de referência de uso da classe
\textsf{abntex2} e do pacote \textsf{abntex2cite}. O documento exemplifica a elaboração de trabalho acadêmico produzido conforme a ABNT NBR 14724:2011 \emph{Informação e documentação - Trabalhos acadêmicos - Apresentação}.

O modelo apresentado é baseado no ``Modelo Canônico'' criado pela equipe do projeto \abnTeX\, e implementa os requisitos das normas da ABNT. Uma lista completa das normas
observadas pelo \abnTeX\ é apresentada em \citeonline{abntex2classe}. Aqui, está apresentada a forma que o modelo será utilizado no curso de Bacharelado em Sistemas de Informação do IFC - Araquari.

Este documento deve ser utilizado como complemento dos manuais do \abnTeX\ 
\cite{abntex2classe,abntex2cite,abntex2cite-alf} e da classe \textsf{memoir}
\cite{memoir}. 

Na introdução o autor coloca o problema ou a indagação que o levou a escrever o texto. A introdução nos dá, então, uma idéia do assunto tratado. Além disso, nela o autor coloca também o ponto de vista ou o ângulo sob o qual ele vai abordar o assunto e, às vezes, o método, ou seja, o caminho que vai seguir (se vai apresentar casos para chegar a uma generalização, ou se vai partir de um princípio geral e deduzir suas consequências).

Também na introdução, o tema é apresentado e esclarecido aos leitores as indicações de leitura do trabalho. Deve-se utilizar o projeto do TCC para colocar na introdução o objetivo principal, os objetivos específicos, o problema e a hipótese.

A respeito de materiais e métodos, pode-se falar sobre a infra-estrutura necessária para o trabalho, incluindo servidores, estações, equipamentos de rede, \emph{softwares} com suas respectivas versões e tudo o mais que for necessário.

A introdução termina com a apresentação dos demais capítulos do trabalho. O capítulo 2 contém o referencial teórico deste trabalho. No capítulo 3 é apresentado o desenvolvimento, o cenário, os testes e a discussão dos resultados. Por fim, temos a conclusão, onde discutiremos os resultados, as dificuldades encontradas e faremos sugestões de trabalhos futuros.